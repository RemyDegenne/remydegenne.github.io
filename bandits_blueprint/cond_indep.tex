% !TeX root = blueprint.tex

\subsection{Standard Borel spaces}
\label{sub:standard_borel_spaces}

Thanks to the ongoing work of Felix Weilacher, we will soon get that every Polish Borel space is measurably equivalent to either a finite set, $\mathbb{Z}$ or $\mathbb{R}$. Such a space is called a standard Borel space.

We need an induction principle to reduce to those three cases.

\begin{lemma}
If a property is true in finite borel spaces, in $\mathbb{Z}$ and in $\mathbb{R}$, and that property is stable by measurable equivalence, then it is true in all standard Borel spaces.
\end{lemma}

It will be easier to have multiple versions of this lemma, depending on the cardinality of the space. Also, using $[0, 1]$ instead of $\mathbb{R}$ might be useful sometimes.

\subsection{Conditional expectation as a kernel}
\label{sub:conditional_expectation_as_a_kernel}


In a standard Borel space, the conditional expectation with respect to a $\sigma$-algebra $\mathcal Q$ defines a kernel from $\Omega$ to $\Omega$, that is a measurable map from $\Omega$ to $\mathcal M(\Omega)$, the measures over $\Omega$.
Indeed, we can verify that $s \mapsto \mathbb{E}[\mathbb{1}_{s} \mid \mathcal Q](\omega)$ defines a measure for all $\omega \in \Omega$, and that the map is measurable.
Let's denote that kernel by $\mathbb{P}_{| \mathcal Q}: \Omega \to \mathcal M(\Omega)$.

TODO: check that the expectation with respect to the measure associated to the conditional expectation is the conditional expectation: $\mathbb{P}_{| \mathcal Q}(\omega)[f] = \mathbb{E}[f \mid \mathcal Q](\omega')$.

If the space is not standard Borel, then the conditional expectation exists and we can talk about $\mathbb{E}[\mathbb{1}_{s} \mid \mathcal Q](\omega)$ for any set $s$ and any $\omega$, but this does not in general define a measure.

TODO: we can define conditional independence in the general setting, but we cannot refactor it as a special case of the kernel notion of independence.

\paragraph{Disintegration of kernels}

TODO: do we really need it? When we build the sequential learning problem, we input kernels for some conditional distributions. We don't need to prove they exist. We can also talk about conditional independence without having kernels. But we might need to consider other conditionals, and might need that they define kernels? Unsure.

Need Radon-Nikodym for kernels. Then the proof is similar to disintegration of measures.

\subsection{Conditional independence}
\label{sub:conditional_independence}


As of the time of writing this document, mathlib contains a definition of independence, but not conditional independence.

\begin{definition}
Two $\sigma$-algebras $\mathcal S, \mathcal T$ are independent given a third $\sigma$-algebra $\mathcal Q$ if for all $s \in \mathcal S$ and $t \in \mathcal T$, almost surely
$
\mathbb{E}[\mathbb{1}_{s \cap t} \mid \mathcal Q] = \mathbb{E}[\mathbb{1}_{s} \mid \mathcal Q] \: \mathbb{E}[\mathbb{1}_{t} \mid \mathcal Q]
\: .
$

That is, almost surely $\mathbb{P}_{| \mathcal Q}(\omega)(s \cap t) = \mathbb{P}_{| \mathcal Q}(\omega)(s) \: \mathbb{P}_{| \mathcal Q}(\omega)(t)$.
\end{definition}

This is close to the definition of (unconditional) independence, which we write below.
\begin{definition}
Two $\sigma$-algebras $\mathcal S, \mathcal T$ are independent if for all $s \in \mathcal S$ and $t \in \mathcal T$,
$
\mathbb{P}(s \cap t) = \mathbb{P}(s) \: \mathbb{P}(t)
\: .
$
\end{definition}

The main difference when we go from independence to conditional independence is that we replace equality of probabilities of sets by almost everywhere equality of kernels (in the standard Borel case). This motivates the following more abstract definition.

\begin{definition}\label{def:independence_wrt_map}
Two $\sigma$-algebras $\mathcal S, \mathcal T$ are independent given a kernel $\kappa : \Omega' \to \mathcal M(\Omega)$ and a measure $\mu'$ on $\Omega'$ if for all $s \in \mathcal S$ and $t \in \mathcal T$, for $\mu'$-almost all $\omega' \in \Omega'$,
\begin{align*}
\kappa (\omega') (s \cap t) = \kappa (\omega') (s) \: \kappa (\omega') (t)
\: .
\end{align*}
\end{definition}

Unconditional independence is obtained by setting $\Omega' = \star$ (a single point), $\mu' = \delta_\star$ and $\kappa : \omega' \mapsto \mathbb{P}$.

Conditional independence is obtained with $\Omega' = \Omega$, $\mu' = \mathbb{P}$ and $\kappa = \mathbb{P}_{| \mathcal Q}$.

Beyond the benefit of allowing us to prove properties of both types of independence at the same time, Definition~\ref{def:independence_wrt_map} simplifies the way proofs about conditional expectations are written in Lean, since they replace complicated expressions of the form $\mathbb{E}[\mathbb{1}_{s} \mid \mathcal Q]$ by $\kappa$.

TODO: there is a very general version of independence of kernels given other kernels, which is slightly stronger than the definitions we have here (but equivalent in standard Borel spaces). See \cite{forre2021transitional}.
See also the categorical generalization of that definition in \cite{fritz2023d}.