% !TeX root = blueprint.tex

\subsection{Conditional expectation as a kernel}
\label{sub:conditional_expectation_as_a_kernel}


The conditional expectation with respect to a $\sigma$-algebra $\mathcal Q$ defines a map $\Omega \to \mathcal M(\Omega)$, where $\mathcal M(\Omega)$ denotes the measures over $\Omega$.
Indeed, we can verify that $s \mapsto \mathbb{E}[\mathbb{1}_{s} \mid \mathcal Q](\omega)$ defines a measure for all $\omega \in \Omega$.

Let's denote that map by $\mathbb{P}_{| \mathcal Q}: \Omega \to \mathcal M(\Omega)$.
That map is even a kernel when $\Omega$ is standard Borel. That is, $\mathbb{P}_{| \mathcal Q}$ is measurable (or there is a measurable version, which means that it is a.e.-measurable?). We might not need that property? Perhaps we do.

TODO: check that the expectation with respect to the measure associated to the conditional expectation is the conditional expectation: $\mathbb{P}_{| \mathcal Q}(\omega)[f] = \mathbb{E}[f \mid \mathcal Q](\omega')$.

\subsection{Conditional independence}
\label{sub:conditional_independence}


As of the time of writing this document, mathlib contains a definition of independence, but not conditional independence.

\begin{definition}
Two $\sigma$-algebras $\mathcal S, \mathcal T$ are independent given a third $\sigma$-algebra $\mathcal Q$ if for all $s \in \mathcal S$ and $t \in \mathcal T$, almost surely
$
\mathbb{E}[\mathbb{1}_{s \cap t} \mid \mathcal Q] = \mathbb{E}[\mathbb{1}_{s} \mid \mathcal Q] \: \mathbb{E}[\mathbb{1}_{t} \mid \mathcal Q]
\: .
$

That is, almost surely $\mathbb{P}_{| \mathcal Q}(\omega)(s \cap t) = \mathbb{P}_{| \mathcal Q}(\omega)(s) \: \mathbb{P}_{| \mathcal Q}(\omega)(t)$.
\end{definition}

This is close to the definition of (unconditional) independence, which we write below.
\begin{definition}
Two $\sigma$-algebras $\mathcal S, \mathcal T$ are independent if for all $s \in \mathcal S$ and $t \in \mathcal T$,
$
\mathbb{P}(s \cap t) = \mathbb{P}(s) \: \mathbb{P}(t)
\: .
$
\end{definition}

The main difference when we go from independence to conditional independence is that we replace equality of probabilities of sets by almost everywhere equality of kernels (or maps to measures). This motivates the following more abstract definition.

\begin{definition}\label{def:independence_wrt_map}
Two $\sigma$-algebras $\mathcal S, \mathcal T$ are independent given a map $\kappa : \Omega' \to \mathcal M(\Omega)$ and a measure $\mu'$ on $\Omega'$ if for all $s \in \mathcal S$ and $t \in \mathcal T$, for $\mu'$-almost all $\omega' \in \Omega'$,
\begin{align*}
\kappa (\omega') (s \cap t) = \kappa (\omega') (s) \: \kappa (\omega') (t)
\: .
\end{align*}
\end{definition}

Unconditional independence is obtained by setting $\Omega' = \star$ (a single point), $\mu' = \delta_\star$ and $\kappa : \omega' \mapsto \mathbb{P}$.

Conditional independence is obtained with $\Omega' = \Omega$, $\mu' = \mathbb{P}$ and $\kappa = \mathbb{P}_{| \mathcal Q}$.

Beyond the benefit of allowing us to prove properties of both types of independence at the same time, Definition~\ref{def:independence_wrt_map} simplifies the way proofs about conditional expectations are written in Lean, since they replace complicated expressions of the form $\mathbb{E}[\mathbb{1}_{s} \mid \mathcal Q]$ by $\kappa$.

TODO: there is a very general version of independence of kernels given other kernels, which is slightly stronger than the definitions we have here (but equivalent in standard Borel spaces). See \cite{forre2021transitional}.
See also the categorical generalization of that definition in \cite{fritz2023d}.